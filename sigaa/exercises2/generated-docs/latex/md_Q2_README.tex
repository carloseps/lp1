{\bfseries{2.\+3)}} Exemplo\+: um dos programadores da equipe realizou alterações no repositório, mas só após realizar o commit percebeu que havia alguns erros que antes não existiam. O comando {\itshape git reset} permite reverter ou desfazer alterações em um repositório. Seja desfazer alterações em um arquivo, reverter commits ou voltar para um commit anterior do histórico.

{\bfseries{2.\+4)}} o H\+E\+AD é um ponteiro especial que representa a versão atual do código-\/fonte em um determinado ramo do repositório e é atualizado automaticamente pelo Git quando novos commits são feitos.

{\bfseries{2.\+6)}} Fork é uma cópia separada de um repositório inteiro. Sendo meio que um projeto igual, mas algo à parte.

Branch é uma ramificação do fluxo de desenvolvimento de um repositório, quando fazemos alterações na branch, não modificamos o main, mas podemos solicitar através do {\itshape git merge}.

{\bfseries{2.\+8)}} Stash é um recurso do Git que salva temporariamente as alterações no repositório, isso permite que você mude de branch ou trabalhe em outra tarefa sem necessariamente commitar incompletamente.

{\bfseries{2.\+12)}} Sim. Exemplo\+: um dos programadores da equipe realizou alterações no repositório, mas só após realizar o commit percebeu que havia alguns erros que antes não existiam. Ele pode resolver esses problemas alterando o commit. Isso é possível graças ao comando \char`\"{}git commit -\/-\/amend\char`\"{}. Geralmente a gente utiliza quando o commit tem um erro ou precisa ser modificado de alguma forma. 